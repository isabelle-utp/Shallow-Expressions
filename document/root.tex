\documentclass[11pt,a4paper]{article}
\usepackage[T1]{fontenc}
\usepackage{isabelle,isabellesym}

% further packages required for unusual symbols (see also
% isabellesym.sty), use only when needed

\usepackage{amssymb}
  %for \<leadsto>, \<box>, \<diamond>, \<sqsupset>, \<mho>, \<Join>,
  %\<lhd>, \<lesssim>, \<greatersim>, \<lessapprox>, \<greaterapprox>,
  %\<triangleq>, \<yen>, \<lozenge>

%\usepackage{eurosym}
  %for \<euro>

\usepackage{stmaryrd}
  %for \<Sqinter>

%\usepackage{eufrak}
  %for \<AA> ... \<ZZ>, \<aa> ... \<zz> (also included in amssymb)

\usepackage{textcomp}
  %for \<onequarter>, \<onehalf>, \<threequarters>, \<degree>, \<cent>,
  %\<currency>

% this should be the last package used
\usepackage{pdfsetup}

% urls in roman style, theory text in math-similar italics
\urlstyle{rm}
\isabellestyle{it}

% for uniform font size
%\renewcommand{\isastyle}{\isastyleminor}


\begin{document}

\title{Shallow Expressions}
\author{Simon Foster \\[.5ex] University of York, UK \\[2ex] \texttt{\small simon.foster@york.ac.uk}}
\maketitle

\begin{abstract}
\noindent Most verification techniques use expressions, for example when assigning to variables or 
forming assertions over the state. Deep embeddings provide such expressions using a datatype, which 
allows queries over the syntax, such as calculating the free variables, and performing substitutions. 
Shallow embeddings, in contrast, model expressions as query functions over the state type, and are 
more amenable to automating verification tasks. The goal of this library is provide an intuitive 
implementation of shallow expressions, which nevertheless provides many of the benefits of a deep 
embedding. We harness the Optics library to provide an algebraic semantics for state variables,
and use syntax translations to provide an intuitive lifted expression syntax. Furthermore, we
provide a variety of meta-logic-style queries on expressions, such as dependencies
on a state variable, and substitution of a variable for an expression. We also provide proof 
methods, based on the simplifier, to automate the associated proof tasks.
\end{abstract}

\tableofcontents

% sane default for proof documents
\parindent 0pt\parskip 0.5ex

\section{Introduction}

This session provides a library for expressions in shallow embeddings, based on the Optics
package~\cite{Optics-AFP}. It provides the following key features:

\begin{enumerate}
  \item Parse and print translations for converting between intuitive expression syntax,
    and state functions using lenses to model variables. The translation uses the type
    system to determine whether each free variable in an expression is (1) a lens (i.e. a state 
    variable); (2) another expression; (3) a literal, and gives the correct interpretation for each.
    The lifting mechanism is manifested through the bracket notation, $(\_)_e$, but can usually be hidden behind syntax.
  \item Syntax for complex state variable constructions, using the lens operators~\cite{Optics-AFP}, such as simultaneous 
    assignment, hierarchical state, and initial/final state variables.
  \item The ``unrestriction'' predicate~\cite{Oliveira07,Foster2020-IsabelleUTP}, $x \mathop{\sharp} e$, 
    which characterises whether an expression $e$ depends on a
    particular variable $x$ or not, based on the lens laws. It can often be used as a replacement for 
    syntactically checking for free variables in a deep embedding, as needed in several verification
    techniques.
  \item Semantic substitution of variables for expressions, $e[v/x]$, with support for evaluation from 
    the simplifier. A notation is provided for expressing substitution objects as a sequence of simultaneous 
    variable assignments, $[x_1 \leadsto e_1, x_2 \leadsto e_2, \cdots]$.
  \item Collection lenses, $x[i]$, which can be used to model updating a component of a larger structure,
    for example mutating an element of an array by its index.
  \item Supporting transformations and constructors for expressions, such as state extension, 
    state restriction, and quantifiers.
\end{enumerate}

The majority of these concepts have been adapted from Isabelle/UTP~\cite{Foster2020-IsabelleUTP}, but 
have been generalised for use in other Isabelle-based verification tools. Several proof methods are 
provided, such as for discharging unrestriction conditions (\texttt{unrest}) and evaluating substitutions (\texttt{subst\_eval}).

The Shallow Expressions library has been applied in the IsaVODEs tool~\cite{Munive2024IsaVODES}, for 
verifying hybrid systems, and Isabelle/ITrees~\cite{Foster2024ITrees}, for verification of process-algebraic languages. 

% generated text of all theories
\input{session}

% optional bibliography
\bibliographystyle{abbrv}
\bibliography{root}

\end{document}

%%% Local Variables:
%%% mode: latex
%%% TeX-master: t
%%% End:
