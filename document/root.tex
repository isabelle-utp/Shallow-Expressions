\documentclass[11pt,a4paper]{article}
\usepackage[T1]{fontenc}
\usepackage{isabelle,isabellesym}

% further packages required for unusual symbols (see also
% isabellesym.sty), use only when needed

\usepackage{amssymb}
  %for \<leadsto>, \<box>, \<diamond>, \<sqsupset>, \<mho>, \<Join>,
  %\<lhd>, \<lesssim>, \<greatersim>, \<lessapprox>, \<greaterapprox>,
  %\<triangleq>, \<yen>, \<lozenge>

%\usepackage{eurosym}
  %for \<euro>

\usepackage{stmaryrd}
  %for \<Sqinter>

%\usepackage{eufrak}
  %for \<AA> ... \<ZZ>, \<aa> ... \<zz> (also included in amssymb)

\usepackage{textcomp}
  %for \<onequarter>, \<onehalf>, \<threequarters>, \<degree>, \<cent>,
  %\<currency>

% this should be the last package used
\usepackage{pdfsetup}

% urls in roman style, theory text in math-similar italics
\urlstyle{rm}
\isabellestyle{it}

% for uniform font size
%\renewcommand{\isastyle}{\isastyleminor}


\begin{document}

\title{Shallow Expressions}
\author{Simon Foster}
\maketitle

\begin{abstract}
\noindent Most verification techniques use expressions, for example when assigning to variables or 
forming assertions over the state. Deep embeddings provide such expressions using a datatype, which 
allows queries over the syntax, such as calculating the free variables, and performing substitutions. 
Shallow embeddings, in contrast, model expressions as query functions over the state type, and are 
more amenable to automating verification tasks. The goal of this library is provide an intuitive 
implementation of shallow expressions, which nevertheless provides many of the benefits of a deep 
embedding. We harness the Optics library to provide an algebraic semantics for state variables,
and use syntax translations to provide an intutive lifted expression syntax. Furthermore, we
provide a variety of meta-logic-style queries on expressions, such as dependencies
on a state variable, and substitution of a variable for an expression. We also provide proof 
methods, based on the simplifier, to automate the associated proof tasks.
\end{abstract}

\tableofcontents

% sane default for proof documents
\parindent 0pt\parskip 0.5ex

% generated text of all theories
\input{session}

% optional bibliography
\bibliographystyle{abbrv}
\bibliography{root}

\end{document}

%%% Local Variables:
%%% mode: latex
%%% TeX-master: t
%%% End:
